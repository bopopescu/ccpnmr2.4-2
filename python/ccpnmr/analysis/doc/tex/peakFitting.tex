\setupoutput[pdftex]

\setupcolors[state=start]

\usemodule[units]

\setuppagenumbering[location=footer]

\setuppagenumbering[state=start]

\setupwhitespace[big]

\setuplayout[leftmargin=1cm, rightmargin=1cm]
\setuplayout[leftedge=0cm, rightedge=0cm]
\setuplayout[header=1cm, footer=1cm]
\setuplayout[top=0cm, bottom=0cm]

\setupheads[alternative=middle]

\starttext
\title{Peak Fitting in Analysis}

\setupheads[alternative=normal]


The Gaussian fitting function in N dimensions is

$$G(x) = h \prod_{i=1}^N e^{-(x_i-c_i)^2/a_i^2}$$

where $h$ is the height (intensity) and $c_i$ is the center and $a_i$ is the ``width''.

The linewidth, $l_i$, is the full width at half max.
We solve $e^{-x^2/a^2} = 1/2$ to find that $x = a \sqrt {ln2}$.  Thus the linewidth
$l = 2a \sqrt {ln2}$ and so $a = l / (2 \sqrt {ln2})$.

In terms of the linewidth we have

$$G(x) = h \prod_{i=1}^N e^{-4 \ ln2 \ (x_i-c_i)^2/l_i^2}$$

The Lorentzian fitting function in N dimensions is

$$L(x) = h \prod_{i=1}^N {a_i^2 \over a_i^2 + (x_i-c_i)^2}$$

To find the linewidth we solve ${a^2 \over a^2 + x^2} = 1/2$ to find that
$x = a$.  Thus the linewidth $l = 2a$ and so $a = l/2$.

In terms of the linewidth we have

$$L(x) = h \prod_{i=1}^N {l_i^2 \over l_i^2 + 4(x_i-c_i)^2}$$

Note that there are 1+2N parameters for the fitting, $(h, c_i, l_i)$.  For nonlinear fitting we need derivatives
of the functions with respect to the parameters.  We have

$$\eqalign{{\partial G \over \partial h} &= {G \over h} \cr
  {\partial G \over \partial c_i} &= {8 \ ln2 \ (x_i - c_i) G \over l_i^2} \cr
  {\partial G \over \partial l_i} &= {8 \ ln2 \ (x_i - c_i)^2 G \over l_i^3} \cr
}$$

and

$$\eqalign{{\partial L \over \partial h} &= {L \over h} \cr
  {\partial L \over \partial c_i} &= {8 (x_i - c_i) L \over l_i^2 + 4(x_i-c_i)^2} \cr
  {\partial L \over \partial l_i} &= {2 L \over l_i} - {2 l_i L \over l_i^2 + 4(x_i-c_i)^2}
    = {8 (x_i - c_i)^2 L \over l_i(l_i^2 + 4(x_i-c_i)^2)}\cr
}$$

The volume of the Gaussian function is

$$\int G(x) = h \prod_{i=1}^N \sqrt \pi a_i = h \prod_{i=1}^N {\sqrt \pi l_i \over 2 \sqrt {ln2}}
  = h \left ( {1 \over 2} \sqrt {\pi \over ln2} \right )^N \prod_{i=1}^N l_i$$

The volume of the Lorentzian function is

$$\int L(x) = h \prod_{i=1}^N \pi a_i = h \prod_{i=1}^N {\pi l_i \over 2}
  = h \left ({\pi \over 2} \right )^N \prod_{i=1}^N l_i$$

For a peak that is fitted with a Gaussian, there is a Lorentzian line shaped peak with the same
volume.  Let $l'_i = s l_i$ be the equivalent Lorentzian linewidth.  Then

$$h \left ( {1 \over 2} \sqrt {\pi \over ln2} \right )^N \prod_{i=1}^N l_i = h \left ({\pi \over 2} \right )^N \prod_{i=1}^N l'_i$$

and so

$${1 \over 2} \sqrt {\pi \over ln2} = s {\pi \over 2}$$

and thus

$$s = {1 \over \sqrt { \pi ln2 }} \approx 0.678$$

Assuming we have a peak with an actual maximum, then the initial values can be determined somewhat
sensibly.  So $h$ can be set to the value of the peak at the maximum position, and $c_i$ can be set
to the maximum (or using a parabolic interpolation to a non-grid point), and $l_i$ can be set by
seeing where the peak decreases to half its value or starts to turn up again (again, interpolation
can be used).

So for isolated peaks, this fitting should work fairly well.

With overlapped peaks you have to do sums of the functions, which starts to give huge numbers of
parameters for fitting, which is not ideal.  It's quite possible that the fitting in this case is
very sensitive.

\stoptext
